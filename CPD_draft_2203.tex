% Please do not change the document class
\documentclass{scrartcl}

% Please do not change these packages
\usepackage[hidelinks]{hyperref}
\usepackage[none]{hyphenat}
\usepackage{setspace}
\doublespace

% You may add additional packages here
\usepackage{amsmath}

% Please include a clear, concise, and descriptive title
\title{CPD Report}

% Please do not change the subtitle
\subtitle{COMP160 - CPD Report}

% Please put your student number in the author field
\author{Student Number: 1607539}

\begin{document}

\maketitle

\section*{The second term}

The group project has taken up a sizeable portion of our focus this term. This contrasts with the first semester where there was an equal focus on all the modules. This has led to other deadlines creeping up on us all. The essay proposal and hardware hacking plans were a little rushed through. I have now collected all the deadlines, large or small, into one concise notepad document - now always open in my desktop.

I can reflect at this point that it would be pertinent to regularly check said notepad document. Again, deadlines have crept up. It has been easier to rally this time, perhaps because of having to have written an essay relatively recently compared to before last semesters deadline, or perhaps because I find the subject of my portability essay more interesting than that of Agile software development. In either case, it is with anticipation that I face the coming research journal task. Despite, and one day perhaps I will learn, having delayed starting it to later than I should have.

Looking back on how I balanced my busy work schedule around the university deadlines changes how I look to plan the next year's balance.  In hindsight this seems obvious, but I plan to take annual leave from work in the week leading to large deadlines, found to be relevant this year with deadlines falling at the end of Easter, and Easter being a particularly stressful time in hospitality.

\section*{The group project}

The largest change in this term has been being thrust into a group of twelve students, mostly strangers to each other, banding together to work on a group project.  Whilst certainly not without fault, I think that we did well.  Aspects in which we did fail should be learnt from.  These centre mostly around our group dynamic; the exclusion of some members following inactivity, the acceptance of others perhaps due to friendships, and the way in which any unproductiveness was  handled.  Without any Agile methodology police, and often high spirits, apparent leaders would rise and fall, and approach any shortfalls quite differently.  I hope that this is looked back upon as a trial and error process for the next group project.  That the methods which worked well and those that didn't, do not need reiterating. We are all relatively intelligent and I have faith that no one will blindly repeat any mistakes.

\section*{The second year}

There was some disruption following the second year team announcements. This has led to all sorts of chaos and chatter, but most noticeable to me was the almost instant disbanding of the first year teams. Knowing that the work accomplished so far was most probably adequate to pass, that the grade of the work will have no impact upon any ultimate grades, and to have confirmation that we would not work together in the second year, led to some voicing their disillusionment and disenfranchisement resignedly. I for one, am excited for the same reasons. That we have created a game to be proud of, that I believe is enough to cement a good first year grade, relieves any pressure to agonize in stressful pre deadline crunch work. Instead I have enjoyed the freedom to explore new areas of Unreal; widgets, animations, and better blueprint practices.  The T-shaped model must be maintained! For the summer, I have asked the members of my team for advice on tutorials in their own fields to complete. It can only be beneficial to have some understanding of Maya for example, having worked alongside artists more so than any other specialist. I have no doubt that were they to have a better understanding of blueprint classes, or I a better understanding of Maya, we would have been even more productive.

Since then, a small number of us have started working on a summer project.  As much an opportunity to learn more, as to practise what we have already learnt.  This may impact on my time to study other areas, but I don't regret this.  It is refreshing to have a project of our own that will not be subjected to gradation.  I plan to reflect with the other disciplines in my team, on what skills would be helpful to them, for me to have.  Then perhaps, I will be in a better position to judge whether to follow through and study them.

\section*{Writing code}

I miss writing code. Blueprints are very intuitive and produce incredible results extremely quickly. They are not too easy to work with, so there is scope to learn and be challenged, but I still miss the satisfaction in writing code. This is an easy problem to address and fix as the solution is simply to write more code. I have since purchased a new C++ book "Beginning C++ Through Game Programming" by Michael Dawson, and have been slowly working through "Programming: Principles and Practice Using C++" by Bjarne Stroustrup. Additionally, our new lecturer Brian McDonald has agreed to assign some extra time, outside of timetabled hours, to dedicate to just writing and learning C++, whatever we need. He has been more than helpful in explaining whatever questions I've come to him with a good level of abstraction.

With my summer project, we have already decided to prototype as far as we can with blueprints for ease, then as an exercise, to rewrite everything from the ground up with C++ classes.  Sharing the same libraries makes this entirely possible and could even allow better options that are not possible in blueprints.

\section*{Where do you see yourself...}

Perhaps as a CPD for my CPD, 'what happens next?' is a question that I must continually be asking myself.  We were all asked recently where we saw ourselves in three years' time.  I immediately said to be completing a masters.  With luck and hard work, I should be eligible so why should or shouldn't I?  After returning to education after working, moving, and paying rent for so long I aim on making full use of every opportunity I can, but am I just anxious about returning tofull time emplyment?  If a good job opportunity were to present itself should I take it?  As it stands, I am just enjoying learning as much as I can, and I hope to do so for as long as I can.  I answered the question so assuredly when it was asked, I have no doubt that it is what I plan to do, but then what comes after that?  I know from experience not to postpone planning for the future.  So, without letting the decision distract me, I must pay some thought as to where this degree is taking me.

Knowing myself, I should try every possible option open to me.  As an indie game developer, there is our summer project we look to take to market by the end of the summer.  As a larger studio developer, possible work placements have already been discussed.  I enjoyed briefly helping with a school visit, so there is the option of teaching, and given the broadly applicable computer skills being taught, I should consider the myriad of options available that aren't in game development.

\end{document}
