\documentclass{scrartcl}

\usepackage[hidelinks]{hyperref}
\usepackage[none]{hyphenat}

\title{CPD Report - Draft}

\author{Student Number: 1607539}

\begin{document}

\maketitle

\section*{Where to focus?}

It has been stressed to us that our primary focus should currently be on the group project. This is in direct contrast to the first semester where there was an equal focus on all the modules. This has led to other deadlines creeping up on us all. The essay proposal and hardware hacking plans were a little rushed through. I have now collected all of the deadlines, large or small, into one concise notepad document - now always open in my desktop.

I can reflect at this point that it would be pertinent to regularly check said notepad document. Again, in the excitement of working on a project that has been previously out of a single developer's scope, with students of many other disciplines, deadlines have sneakily crept up. It has been easier to rally this time, perhaps because of having to have written in an academic style relatively recently compared to before last semesters tasks, or perhaps because I find the subject of portability far more interesting than that of Agile software development for an essay subject. In either case, it is with refreshing anticipation rather than looming dread that I face the coming research journal task. Despite, and one day perhaps I will learn, having delayed starting it to later than I should have.

\section*{The group project}

The largest change this semester has been being thrust into a group of twelve students, mostly strangers to each other, banding together to work on a group project. Perhaps it is just pride, but I believe that we are the strongest team. After hearing stories of other teams barely leaving the drawing board, with team members rarely if ever attending stand ups, I feel very lucky to be a part of this team.

Not that it is completely without fault, there have been issues with attendance, productivity, and direction. It can be difficult to address this within Agile practices as there is no clearly defined police. The main point of authority is the scrum master obviously, however their job is not to manage, but to aid adherence to scrum mentalities etc. Not to mention the large workload that any member of our group has already. To address these issues, we have tried group meetings, peer 'encouragement' (artist to artist, designer to designer), scrum master involvement, and project owner involvement. There have been some improvements but I am still unsure how to properly approach this in the future, or even whose responsibility it should be?

\section*{The second year}

There was some disruption following the second year team announcements. This has led to all sorts of chaos and chatter, but most noticeable to me was the almost instant disbanding of the first year teams. Knowing that the work accomplished so far was most probably adequate to pass, that the grade of the work will have no impact upon any ultimate grades, and to have confirmation that we would not work together in the second year, led to some voicing their disillusionment and disenfranchisement resignedly. I for one, am excited for the same reasons. That we have created a game to be proud of, that I believe is enough to cement a good first year grade, relieves any pressure to agonize in stressful pre deadline crunch work. Instead I have enjoyed the freedom to explore new areas of Unreal; widgets, animations, and better blueprint practices. I know I can write a program, I have the potential to be a good programmer even. But I would like to be 'full stack' as far as I can. The T-shaped model must be maintained! For the summer, I have asked the members of my team for advice on tutorials in their own fields to complete. It can only be beneficial to have some understanding of Maya for example, having worked alongside artists more so than any other specialist. I have no doubt that were they to have a better understanding of blueprint classes, or I a better understanding of Maya, we would have been even more productive.

\section*{Writing code}

I miss writing code. Blueprints are very intuitive and produce incredible results extremely quickly. They are not too easy to work with, so there is scope to learn and be challenged, but I still miss the satisfaction in writing code. This is an easy problem to address and fix as the solution is simply to write more code. I have since purchased a new C++ book "Beginning C++ Through Game Programming" by Michael Dawson, and have been slowly working through "Programming: Principles and Practice Using C++" by Bjarne Stroustrup, as well as phone apps, and my worksheets. For instance, I intend to complete the worksheet C initially in blueprints, and then again in C++. Additionally, our new lecturer Brian McDonald has agreed to assign some extra time, outside of timetabled hours, to dedicate to just writing and learning C++, whatever we need. He has been more than helpful in explaining whatever questions I've come to him with with a good level of abstraction.

Whilst looking through the academy's GitHub repositories I have found what I think is last year's worksheets. Pathfinding in particular sounds very interesting, and useful; in fact, essential in game development. This should last me through the summer. I think that having worksheet challenges to follow suits me more than an arbitrary personal project.

\section*{Where do you see yourself...}

Perhaps as a CPD for my CPD, 'what happens next?' is a question that I must continually be asking myself. We were all asked recently where we saw ourselves in three year's time. I immediately said to be completing a masters. With luck and hard work, I should be eligible so why should or shouldn't I? After returning to education after working, moving, and paying rent for so long I promised myself that I would 'whole ass' this one thing and make full use of every opportunity I could, but am I just anxious about returning to the rat race? If a good job opportunity were to present itself should I take it? As it stands, I am just enjoying learning as much as I can, and I hope to do so for as long as I can. I answered the question so assuredly when it was asked, I have no doubt that it is what I plan to do, but then what comes after that? I know from experience not to postpone answering questions about the future otherwise the future gets postponed. So, without letting the decision distract me, I must pay some thought as to where this degree is taking me.

\end{document}
