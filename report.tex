% Please do not change the document class
\documentclass{scrartcl}

% Please do not change these packages
\usepackage[hidelinks]{hyperref}
\usepackage[none]{hyphenat}
\usepackage{setspace}
\doublespace

% You may add additional packages here
\usepackage{amsmath}

% Please include a clear, concise, and descriptive title
\title{Your Title Here}

% Please do not change the subtitle
\subtitle{COMP150 - CPD Report}

% Please put your student number in the author field
\author{Student Number: 1607539}

\begin{document}

\maketitle

\section{Introduction}

I have been fortunate enough to have had some programming experience before the start of this course and unfortunate enough. My experience has given me a slight head start, a pragmatic and informed approach to writing code although undisciplined and with many bad habits that have needed overcoming. The only problems I have experienced outside of practical knowledge have been related to reintegrating back into a learning environment. Specifically, academic writing and reading, and adjusting the balance of my priorities. My goal is to become an expert at programming and writing code, and by addressing these shortfalls I can better thrive in an academic environment ultimately learning more, and easier.

\section{Academic writing}

It is either that I am very good at writing, or very bad. Whichever it is I do know that my writing style is not suited to academia. After being shown expresso-app.org I was shocked with the amount of unnecessary words that I used not to mention the self-evident, extremely long sentences that I write never knowing really when to stop... I tend to write several drafts of any writing required of me. The first is eloquent, complex, and ultimately pointless. My first research journal draft had five hundred words of digression arguing about the nature of abstract methods with one citation only. Pointless. This takes up far more of my time than it should, and whilst I enjoy expressive writing I need to learn how to write more objectively. Whilst I have read many guides on how to write in an academic fashion, ultimately only through practise will I improve this skill. I ambivalently both dread the time I will need for the next writing assignment, and look forward to strengthening what is ultimately an important skill.

\section{Academic reading}

Like many others I struggled with staying up to date with assigned research journal reading. Unless it interests me - some really do. Peculiarly, ultimately they all interested me, All of the assigned readings, once I had finished reading them, made me want to read them again! I enjoyed following citations through subsequent papers and building a larger picture on a subject, like building a jigsaw. It is the first stage of initially sitting down, putting aside some time, and getting myself into the mental state needed to absorb the information that I have found difficult. For example, the research journal assignment would have been much easier had I kept up with the reading assignments, and not crammed them, and the other eighteen, into the weekend before the due date. As I found satisfaction in being led down a thought path that is unfamiliar to me, I plan to set myself my own research assignment; to at minimum read one research paper a week whether assigned or not. The next research journal should be significantly easier and I may discover an aspect to the field, which I would not have otherwise found, which fascinates me further. One a week, at the least.

\section{Managing my job}

With many financial commitments, I have had to continue in my previous job in a restaurant/bar/boutique hotel that I have worked within for over five years. The last three of which have been as the restaurant manager which, as a privately owned business, holds far more responsibilty than is reasonable. Stepping down to part time, and being able to pick and choose my shifts to fit around my studies. was bargained with retaining some responsibility which has at times had an impact on my studies. For example, as I write this I am aware that I shall be working this evening until about 01:30am. I shall still be in Falmouth ready to study at 9am the following morning, although of course this will have some impact on my concentration. As time has gone on my workplace has become less dependent on me and I have found it easier to readjust to my new lesser role. This has also been a part of the problem. It can be difficult at times to see something happen which would have once been my responsibility to deal with, and to leave it to someone else. Gradually the balance is shifting in my expectations, and in what is expected of me, and will continue to do so. I have been asked to return to a higher level of management over the Christmas holidays after which I plan on meeting with my manager to discuss in absolute terms what my job entails. Now I am starting to understand the demands of higher education I can better state what needs I have.

\section{Learning how I learn}

It is a fact that I have not been in the education system for a long time now, and one problem I have had is in rediscovering how I learn. I am aware that there are many ways in which different people learn different things, perhaps I am a little too set in my ways for some methods, perhaps I now possess some maturity and patience which my eighteen-year-old self did not. For example, a teacher could explain a method to me many times but I would not learn the skill until I had completed the activity myself, or written it down, even just once was enough to concrete the information in my mind. However, lately I have found a new appreciation of lectures for instance. Perhaps It is because I am learning a subject that I have a genuine passion for, perhaps I am more susceptible to this style of learning now. Either way, I am now far more open to the idea of not just having to experience the lesson myself. Whilst still apprehensive of a non-experiential learning situation, I look forward to being proven wrong over the period of this course. To discover new ways to learn - guest speakers, research papers, even from my own research.

\section{Not time management}

I should stress that this section is not about poor time management, just the opposite. I have been keenly dedicated to my studies, `you have to play the game'. If I don't put in the work then I won't get the result that I want. However, I don't think I have given any priority to maintaining any balance. For example, whilst it seems strange to consider playing games an integral part of my studies, game development is a creative process and I must give some thought as to how to maintain some level of inspiration. I have bought many games over the last two months yet have probably played less than two hours in total. In a very quickly changing field I am feeling out of touch with current trends. I have made efforts elsewhere, reading for example, and my creativity has been reinvigorated lately after reading 1984 by George Orwell, albeit in a dystopian way. I have since started Fahrenheit 451 so this train of thought may continue! This reinforces the obvious, that creativity and inspiration have to fed by a balance between `work and play'. I need to focus more on maintaining this balance, especially given that my studies will only become more difficult. This in turn relates to course specific extra work. My surfing game that I proudly worked on for over a year has remained untouched since starting at Falmouth and I need to make time for relevant extra-curricular activities.

\section{Conclusion}

Succeeding in university requires a different approach to that which I have experienced in the working world, and certainly a different skillset to that which I currently possess. I plan on not putting off projects which fall out of my comfort range as only through careful planning, and reinforced by good practice, can I strengthen the skills in which I am lacking. 

\bibliographystyle{ieeetran}
\bibliography{references}

\end{document}

There is only one outcome to this degree that I wish, and that is to become an expert at programming and writing code. To understand and be able to use computers better than the majority of my peers. Whilst I am certain that in my capacity I have some truly great games, I am deterred by unhappy stories from the industry. Even from successes I hear that the hours are long and the pay is short. But to be an expert may excuse me from these shortfalls, or subsequently allow the pursuit of a career that requires the expertise attained; teaching or research. I seriously doubt that I have what it takes to be an expert, but only an expert would really know.

